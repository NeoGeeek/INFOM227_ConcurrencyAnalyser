\section{Domaines abstraits}

\subsection{Domaine abstrait des valeurs}

Notre domaine abstrait des valeurs se compose de cinq éléments :
\begin{center}
    \texttt{\{Used (Top), Read+Write, Read, Write, Free (Bottom)\}}.
\end{center}


Ces cinq éléments représentent les différents états d'une variable pour un accès concurrent. Nous avons fait le choix d'utiliser un cinquième état, \texttt{Read+Write}, pour permettre plus de précision dans notre outil d'analyse.

L'élément Top est \texttt{Used}, qui indique qu'accéder à cette variable est potentiellement problématique. Cela signifie qu'il y a au moins un accès en écriture concurrent à cette variable.\\
L'élément Bottom est \texttt{Free}, qui indique qu'il n'y a aucun accès concurrent à cette variable.\\
L'élément \texttt{Read} indique qu'il y a au moins un accès en lecture concurrent à cette variable, mais aucun accès en écriture.\\
L'élément \texttt{Write} indique qu'il y a au moins un accès en écriture concurrent à cette variable, mais aucun accès en lecture.\\
L'élément \texttt{Read+Write} indique qu'il y a au moins un accès en lecture et un accès en écriture concurrent à cette variable.

\subsection{Lattice}

\begin{figure}[h!]
    \centering
    \includegraphics[width=0.3\textwidth]{figure/lattice}
    \label{fig:Lattice}
\end{figure}

Vous trouvez ci-dessus la figure de notre lattice représentant notre domaine abstrait des valeurs.
Nous avons choisi cette structure car cela permet d'avoir une hiérarchie. Cette hiérarchie nous permet d'avoir des accès uniquement en écriture ou uniquement en lecture.
Lorsqu'un accès en lecture et un accès en écriture sont détectés, nous pouvons remonter dans le lattice jusqu'à l'élément \texttt{Read+Write}, qui est plus précis que l'élément Top, \texttt{Used}.
L'élément Top est atteint uniquement lorsque deux accès en écriture sont détectés concurremment, ce qui est le cas le plus problématique.