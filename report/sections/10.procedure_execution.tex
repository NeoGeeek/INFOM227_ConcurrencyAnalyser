\section{Procédure d'exécution de l'outil d'analyse}

Pour exécuter notre outil d'analyse statique, suivez les étapes ci-dessous :
\begin{enumerate}
    \item \textbf{Pré-requis :} Assurez-vous d'avoir Python 3 installé sur votre machine. Vous pouvez vérifier cela en exécutant la commande suivante dans votre terminal :
    \begin{lstlisting}[style=SmallLang]
        python --version
    \end{lstlisting}

    \item \textbf{Lancement de l'outil :} Assurez-vous d'être dans le répertoire racine du projet.
    Exécutez la commande suivante pour lancer l'outil d'analyse en fournissant le chemin vers le fichier Small à analyser :
    \begin{lstlisting}[style=SmallLang]
        python tool/cli.py <chemin_vers_fichier_small>
    \end{lstlisting}
    Remplacez \texttt{<chemin\_vers\_fichier\_small>} par le chemin réel du fichier Small que vous souhaitez analyser.
    Vous pourrez également utiliser les fichiers d'exemple situés dans le répertoire \texttt{data/}.

    \item \textbf{Interprétation des résultats :} Après l'exécution de l'outil, les résultats de l'analyse seront affichés dans le terminal.
    Si des accès concurrents problématiques sont détectés, ils seront listés avec des détails tels que la variable concernée, les lignes de code impliquées, et le type d'accès (lecture ou écriture).

    \item \textbf{Exemple de lancement :} Voici un exemple de commande pour analyser un fichier nommé \texttt{GenKill\_example.small} situé dans le répertoire \texttt{data/} :
    \begin{lstlisting}[style=SmallLang]
        python tool/cli.py data/GenKill_example.small
    \end{lstlisting}

\end{enumerate}