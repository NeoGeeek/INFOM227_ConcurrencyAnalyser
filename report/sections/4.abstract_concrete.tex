\section{Fonction d'abstraction et de concrétisation}

\subsection{Fonction d'abstraction}
\[
    \alpha(x) =
    \begin{cases}
        \text{Free} & \text{si } x = \emptyset,\\[2mm]
        \text{Read} & \text{si } x \text{ contient uniquement des instructions \textit{read}},\\[1mm]
        \text{Write} & \text{si } x \text{ contient uniquement des instructions \textit{write}},\\[1mm]
        \text{Read+Write} & \text{si } x \text{ contient au moins un \textit{read} et au moins un \textit{write}},\\[1mm]
        \text{Used} & \text{sinon}.
    \end{cases}
\]

\textit{x} est un ensemble d'instructions accédant à une variable.\\
Une instruction est considérée comme un accès \textit{write}, si c'est une assignation de valeur à une variable.\\
Toute autre instruction qui utilise la variable est considérée comme un accès \textit{read}.

\subsection{Fonction de concrétisation}
\[
    \gamma(x) =
    \begin{cases}
        \emptyset & \text{si } x = \text{Free},\\[2mm]
        \text{Uniquement des instructions \textit{read}} & \text{si } x = \text{Read},\\[1mm]
        \text{Uniquement des instructions \textit{write}} & \text{si } x = \text{Write},\\[1mm]
        \text{Au moins une instruction \textit{read} et au moins une instruction \textit{write}} & \text{si } x = \text{Read+Write},\\[1mm]
        \text{Used} & \text{sinon}.
    \end{cases}
\]