\section{Erreurs détectées}

Cette section présente les différentes erreurs et avertissements que notre outil d'analyse statique peut détecter dans les programmes écrits en langage Small.

\subsection{Erreurs d'accès problématiques}
Notre outil est capable de détecter les accès concurrents problématiques aux variables, notamment :
\begin{itemize}
    \item \textbf{Accès simultané en écriture :} lorsque deux threads ou plus tentent d'écrire simultanément dans la même variable.
    \item \textbf{Accès en lecture et écriture concurrents :} lorsqu'un thread lit une variable pendant qu'un autre thread écrit dans cette même variable.
\end{itemize}

\subsection{Avertissements sur les accès potentiellement problématiques}
Notre analyse produit des avertissements lorsque l'outil détecte l'une des erreurs citées précédemment.

Un message apparaît en fin d'analyse pour détailler les accès problématiques détectés.
Le format du message est le suivant :
\begin{center}
    \parbox{0.65\textwidth}{%
        X race candidate(s) found:

        [RACE] var=Y @ line Z (AccessType\_1 vs AccessType\_2) \\
        A: main:AccessType\_1 at line Z \\
        B: lines \{P\} in spawn method\_name(...) in main (spawn line Q)
    }
\end{center}

où :
\begin{itemize}
    \item \textbf{X} : nombre total d'accès problématiques détectés.
    \item \textbf{var=Y} : la variable concernée par l'accès problématique.
    \item \textbf{line Z} : le numéro de ligne où l'accès problématique a été détecté.
    \item \textbf{(AccessType\_1 vs AccessType\_2)} : le type d'accès en conflit (R pour Read, W pour Write, RW pour Read+Write, T pour Thread).
    \item \textbf{lines \{P\}} : les numéros de lignes dans la méthode associée au spawn où l'accès problématique est en conflit.
    \item \textbf{main (spawn line Q)} : le numéro de ligne où l'instruction spaw
\end{itemize}