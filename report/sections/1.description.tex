\section{Description de l'outil d'analyse}

Tout d'abord, nous avons décidé de nous pencher sur le sujet de la concurrence dans les programmes.
Cela permet, dans le langage Small, d'exécuter des portions de code en parallèle via des threads.\\

\subsection{Objectifs}

Plus précisément, notre outil d'analyse statique vise à :

\begin{enumerate}
    \item Vérifier qu’il n’y a pas d’accès simultané en écriture sur une même variable.

    \item Identifier les accès concurrents à une variable, qu’ils soient en lecture ou en écriture.

    \item Fournir des informations précises pour chaque accès problématique détecté, incluant le numéro de ligne ainsi que les threads concernés.
\end{enumerate}

\subsection{Caractéristiques}

\begin{enumerate}
    \item Outil d'analyse statique pour le langage Small, développé en Python.

    \item \textbf{Support d'une concurrence minimale}
    \begin{enumerate}
        \item Création de threads via l'instruction \texttt{spawn <fonction>}.

        \item Synchronisation et unification des états via l'instruction \texttt{await <thread>}.

        \item Les mécanismes plus avancés, tels que la notification ou le blocage explicite, ne sont pas pris en compte afin de conserver un modèle minimaliste.
    \end{enumerate}

    \item \textbf{Analyse multi-variant}
    \begin{enumerate}
        \item Chaque thread possède sa propre abstraction.

        \item L'unification des états possibles s'effectue lors de l'instruction \texttt{await}.
    \end{enumerate}

    \item \textbf{Analyse flow-sensitive}
    \begin{enumerate}
        \item Les états des variables sont suivis instruction par instruction, ce qui permet de détecter les usages concurrents à chaque point du programme.
    \end{enumerate}
\end{enumerate}
