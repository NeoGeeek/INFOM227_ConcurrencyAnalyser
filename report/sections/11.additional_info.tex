\section{Informations supplémentaires}

\subsection{Détail de l'analyseur}

Lors de la création de notre analyseur, nous avons fait plusieurs choix simplifiant notre analyse.
Ces changements nous ont ainsi permis de faciliter l'implémentation d'une analyse statique de code au sein de notre analyseur.

Premièrement, nous avons fait le choix de retirer le KILL du GEN/KILL, qui n'avait d'effet qu'en cas d'un \texttt{await}.
Cela nous a permis de grandement simplifier le concept d'effet d'une fonction ou d'un \texttt{spawn}, tout en l'élargissant à l'ensemble des lignes de code du programme.
Il était donc possible, avec cette simplification, de faire en sorte que notre analyse revienne à une jonction de type \textit{least upper bound} entre les effets des lignes de code.

Ensuite, nous avons fait le choix de grouper les lignes de code en blocs, délimités entre le début et la fin de vie d'un thread.
Cela nous a permis de réduire la détection de \textit{race conditions} à une simple comparaison entre deux effets de blocs.

Enfin, bien que l'analyse implémentée soit une version simplifiée de celle décrite plus haut, elle en garde tout de même les principales propriétés.
En effet, elle reste une analyse \textit{forward} et \textit{flow-sensitive} de type \textit{may}, tout en gardant la sur-approximation des cas problématiques.
Ainsi, bien que moins précise que l'originale, cette implémentation reste \textit{sound}.